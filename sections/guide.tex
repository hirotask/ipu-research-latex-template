%! TEX root = ../main.tex
\documentclass[main]{subfiles}

\begin{document}
\section{作成のガイド}
\subsection{手順}
まず,教務委員会のホームページから,
「Latex要旨作成セット」をダウンロードする.
セットの中身は,スタイルファイル(abstract.sty)とサンプルファイル(sample.tex)と
PDFファイル(sample.pdf)から構成されている.サンプルファイルをそのまま編集して要旨を作成すれば効率がよい.
\subsection{スタイルファイルの使い方}
\begin{verbatim}
 提供するスタイルファイルは,ドキュメントクラスのjarticleを利用しているので,次のように宣言すること.
\documentclass{jarticle}
\usepackage{abstract}
タイトルには,以下の内容を記述する.
\title{日本語の表題}
\etitle{英語の表題}
\author{著者}
\studentid{学籍番号(半角)}
\teacher{
\name{指導教員}
}
\begin{document}
\maketitile
 ・・・本文・・・
副指導教員を記載する(指導教員が複数の)場合は,\teacherの中に以下のように\andをはさんで\nameを追加すればよい.
\teacher{
\name{主指導教員}
\and
\name{副指導教員1}
・・・
}
本文は,基本的に節で区切ること.節の区切りは下記の3つを用いる.
\section{節}
・・・
\subsection{小節}
・・・
\subsection{小々節}
・・・
参考文献については,例えばthebibliography環境を用いて,
\begin{thebibliography}{9}
\bibitem{参照名}
文献1
\bibitem{参照名2}
文献2・・・・
\end{thebibliography}
と記述する.また,本文中には,\cite{参照名}で引用個所を示す.
\end{verbatim}
\end{document}
